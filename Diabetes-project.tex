% Options for packages loaded elsewhere
\PassOptionsToPackage{unicode}{hyperref}
\PassOptionsToPackage{hyphens}{url}
%
\documentclass[
]{article}
\usepackage{amsmath,amssymb}
\usepackage{iftex}
\ifPDFTeX
  \usepackage[T1]{fontenc}
  \usepackage[utf8]{inputenc}
  \usepackage{textcomp} % provide euro and other symbols
\else % if luatex or xetex
  \usepackage{unicode-math} % this also loads fontspec
  \defaultfontfeatures{Scale=MatchLowercase}
  \defaultfontfeatures[\rmfamily]{Ligatures=TeX,Scale=1}
\fi
\usepackage{lmodern}
\ifPDFTeX\else
  % xetex/luatex font selection
\fi
% Use upquote if available, for straight quotes in verbatim environments
\IfFileExists{upquote.sty}{\usepackage{upquote}}{}
\IfFileExists{microtype.sty}{% use microtype if available
  \usepackage[]{microtype}
  \UseMicrotypeSet[protrusion]{basicmath} % disable protrusion for tt fonts
}{}
\makeatletter
\@ifundefined{KOMAClassName}{% if non-KOMA class
  \IfFileExists{parskip.sty}{%
    \usepackage{parskip}
  }{% else
    \setlength{\parindent}{0pt}
    \setlength{\parskip}{6pt plus 2pt minus 1pt}}
}{% if KOMA class
  \KOMAoptions{parskip=half}}
\makeatother
\usepackage{xcolor}
\usepackage[margin=1in]{geometry}
\usepackage{color}
\usepackage{fancyvrb}
\newcommand{\VerbBar}{|}
\newcommand{\VERB}{\Verb[commandchars=\\\{\}]}
\DefineVerbatimEnvironment{Highlighting}{Verbatim}{commandchars=\\\{\}}
% Add ',fontsize=\small' for more characters per line
\usepackage{framed}
\definecolor{shadecolor}{RGB}{248,248,248}
\newenvironment{Shaded}{\begin{snugshade}}{\end{snugshade}}
\newcommand{\AlertTok}[1]{\textcolor[rgb]{0.94,0.16,0.16}{#1}}
\newcommand{\AnnotationTok}[1]{\textcolor[rgb]{0.56,0.35,0.01}{\textbf{\textit{#1}}}}
\newcommand{\AttributeTok}[1]{\textcolor[rgb]{0.13,0.29,0.53}{#1}}
\newcommand{\BaseNTok}[1]{\textcolor[rgb]{0.00,0.00,0.81}{#1}}
\newcommand{\BuiltInTok}[1]{#1}
\newcommand{\CharTok}[1]{\textcolor[rgb]{0.31,0.60,0.02}{#1}}
\newcommand{\CommentTok}[1]{\textcolor[rgb]{0.56,0.35,0.01}{\textit{#1}}}
\newcommand{\CommentVarTok}[1]{\textcolor[rgb]{0.56,0.35,0.01}{\textbf{\textit{#1}}}}
\newcommand{\ConstantTok}[1]{\textcolor[rgb]{0.56,0.35,0.01}{#1}}
\newcommand{\ControlFlowTok}[1]{\textcolor[rgb]{0.13,0.29,0.53}{\textbf{#1}}}
\newcommand{\DataTypeTok}[1]{\textcolor[rgb]{0.13,0.29,0.53}{#1}}
\newcommand{\DecValTok}[1]{\textcolor[rgb]{0.00,0.00,0.81}{#1}}
\newcommand{\DocumentationTok}[1]{\textcolor[rgb]{0.56,0.35,0.01}{\textbf{\textit{#1}}}}
\newcommand{\ErrorTok}[1]{\textcolor[rgb]{0.64,0.00,0.00}{\textbf{#1}}}
\newcommand{\ExtensionTok}[1]{#1}
\newcommand{\FloatTok}[1]{\textcolor[rgb]{0.00,0.00,0.81}{#1}}
\newcommand{\FunctionTok}[1]{\textcolor[rgb]{0.13,0.29,0.53}{\textbf{#1}}}
\newcommand{\ImportTok}[1]{#1}
\newcommand{\InformationTok}[1]{\textcolor[rgb]{0.56,0.35,0.01}{\textbf{\textit{#1}}}}
\newcommand{\KeywordTok}[1]{\textcolor[rgb]{0.13,0.29,0.53}{\textbf{#1}}}
\newcommand{\NormalTok}[1]{#1}
\newcommand{\OperatorTok}[1]{\textcolor[rgb]{0.81,0.36,0.00}{\textbf{#1}}}
\newcommand{\OtherTok}[1]{\textcolor[rgb]{0.56,0.35,0.01}{#1}}
\newcommand{\PreprocessorTok}[1]{\textcolor[rgb]{0.56,0.35,0.01}{\textit{#1}}}
\newcommand{\RegionMarkerTok}[1]{#1}
\newcommand{\SpecialCharTok}[1]{\textcolor[rgb]{0.81,0.36,0.00}{\textbf{#1}}}
\newcommand{\SpecialStringTok}[1]{\textcolor[rgb]{0.31,0.60,0.02}{#1}}
\newcommand{\StringTok}[1]{\textcolor[rgb]{0.31,0.60,0.02}{#1}}
\newcommand{\VariableTok}[1]{\textcolor[rgb]{0.00,0.00,0.00}{#1}}
\newcommand{\VerbatimStringTok}[1]{\textcolor[rgb]{0.31,0.60,0.02}{#1}}
\newcommand{\WarningTok}[1]{\textcolor[rgb]{0.56,0.35,0.01}{\textbf{\textit{#1}}}}
\usepackage{graphicx}
\makeatletter
\def\maxwidth{\ifdim\Gin@nat@width>\linewidth\linewidth\else\Gin@nat@width\fi}
\def\maxheight{\ifdim\Gin@nat@height>\textheight\textheight\else\Gin@nat@height\fi}
\makeatother
% Scale images if necessary, so that they will not overflow the page
% margins by default, and it is still possible to overwrite the defaults
% using explicit options in \includegraphics[width, height, ...]{}
\setkeys{Gin}{width=\maxwidth,height=\maxheight,keepaspectratio}
% Set default figure placement to htbp
\makeatletter
\def\fps@figure{htbp}
\makeatother
\setlength{\emergencystretch}{3em} % prevent overfull lines
\providecommand{\tightlist}{%
  \setlength{\itemsep}{0pt}\setlength{\parskip}{0pt}}
\setcounter{secnumdepth}{-\maxdimen} % remove section numbering
\ifLuaTeX
  \usepackage{selnolig}  % disable illegal ligatures
\fi
\IfFileExists{bookmark.sty}{\usepackage{bookmark}}{\usepackage{hyperref}}
\IfFileExists{xurl.sty}{\usepackage{xurl}}{} % add URL line breaks if available
\urlstyle{same}
\hypersetup{
  pdftitle={Extensive Guide for Health Care Data Analysis using R(Machine Learning Algorithms, GLM)},
  pdfauthor={ERICK@},
  hidelinks,
  pdfcreator={LaTeX via pandoc}}

\title{Extensive Guide for Health Care Data Analysis using R(Machine
Learning Algorithms, GLM)}
\author{ERICK@}
\date{2024-04}

\begin{document}
\maketitle

\hypertarget{this-article-is-all-about-detailed-base-model-analysis-of-the-diabetes-data-which-includes-the-following-analysis}{%
\subsection{\texorpdfstring{\textbf{This article is all about detailed
Base Model analysis of the Diabetes Data which includes the following
analysis:}}{This article is all about detailed Base Model analysis of the Diabetes Data which includes the following analysis:}}\label{this-article-is-all-about-detailed-base-model-analysis-of-the-diabetes-data-which-includes-the-following-analysis}}

\begin{enumerate}
\def\labelenumi{\arabic{enumi}.}
\item
  Data exploration (Data distribution inferences, Univariate Data
  analysis, Two-sample t-test)
\item
  Data Correlation Analysis
\item
  Feature Selection (using Logistic regression)
\item
  Outlier Detection (using principal component graph)
\item
  Basic Parameter Tuning(CV, complexity parameter)
\item
  Data modeling
\end{enumerate}

Basic GLM (With all Features and eliminating few features based on AIC)

\begin{itemize}
\item
  Logistic Regression
\item
  Decision Tree
\item
  Naïve Bayes
\end{itemize}

\begin{Shaded}
\begin{Highlighting}[]
\CommentTok{\#import Libraries}
\FunctionTok{library}\NormalTok{(bookdown)}
\FunctionTok{library}\NormalTok{(tidyverse)}
\FunctionTok{library}\NormalTok{(rmarkdown)}
\FunctionTok{library}\NormalTok{(flexdashboard)}
\end{Highlighting}
\end{Shaded}

\hypertarget{section}{%
\subsection{}\label{section}}

\begin{Shaded}
\begin{Highlighting}[]
\CommentTok{\# \{r warnings=F, message=F\}}
\CommentTok{\#IMPORT DATASET}
\NormalTok{diabetes }\OtherTok{\textless{}{-}} \FunctionTok{read\_csv}\NormalTok{(}\StringTok{"C:/Users/langa/OneDrive/Desktop/R PROGRAMMING PRACTICE/Extensive Guide for Health Care Data Analysis using R(Machine Learning Algorithms)\#/diabetes.csv"}\NormalTok{)}
\NormalTok{diabetes }\OtherTok{\textless{}{-}}\NormalTok{ diabetes }\SpecialCharTok{\%\textgreater{}\%} 
     \FunctionTok{mutate}\NormalTok{(}
       \AttributeTok{Outcome=} \FunctionTok{ifelse}\NormalTok{(Outcome}\SpecialCharTok{==}\DecValTok{0}\NormalTok{, }\StringTok{\textquotesingle{}NO\textquotesingle{}}\NormalTok{,}\StringTok{\textquotesingle{}YES\textquotesingle{}}\NormalTok{), }
                    \CommentTok{\# base::as.factor(Outcome)}
\NormalTok{                           )}

\FunctionTok{head}\NormalTok{(diabetes)}
\end{Highlighting}
\end{Shaded}

\begin{verbatim}
## # A tibble: 6 x 9
##   Pregnancies Glucose BloodPressure SkinThickness Insulin   BMI
##         <dbl>   <dbl>         <dbl>         <dbl>   <dbl> <dbl>
## 1           6     148            72            35       0  33.6
## 2           1      85            66            29       0  26.6
## 3           8     183            64             0       0  23.3
## 4           1      89            66            23      94  28.1
## 5           0     137            40            35     168  43.1
## 6           5     116            74             0       0  25.6
## # i 3 more variables: DiabetesPedigreeFunction <dbl>, Age <dbl>, Outcome <chr>
\end{verbatim}

\begin{Shaded}
\begin{Highlighting}[]
\NormalTok{diabetes}\SpecialCharTok{$}\NormalTok{Outcome }\OtherTok{\textless{}{-}} \FunctionTok{as.factor}\NormalTok{(diabetes}\SpecialCharTok{$}\NormalTok{Outcome)}
\CommentTok{\# levels(diabetes$Outcome)}
\end{Highlighting}
\end{Shaded}

\hypertarget{basic-eda}{%
\subsection{Basic EDA}\label{basic-eda}}

\begin{Shaded}
\begin{Highlighting}[]
\FunctionTok{summary}\NormalTok{(diabetes)}
\end{Highlighting}
\end{Shaded}

\begin{verbatim}
##   Pregnancies        Glucose      BloodPressure    SkinThickness  
##  Min.   : 0.000   Min.   :  0.0   Min.   :  0.00   Min.   : 0.00  
##  1st Qu.: 1.000   1st Qu.: 99.0   1st Qu.: 62.00   1st Qu.: 0.00  
##  Median : 3.000   Median :117.0   Median : 72.00   Median :23.00  
##  Mean   : 3.845   Mean   :120.9   Mean   : 69.11   Mean   :20.54  
##  3rd Qu.: 6.000   3rd Qu.:140.2   3rd Qu.: 80.00   3rd Qu.:32.00  
##  Max.   :17.000   Max.   :199.0   Max.   :122.00   Max.   :99.00  
##     Insulin           BMI        DiabetesPedigreeFunction      Age       
##  Min.   :  0.0   Min.   : 0.00   Min.   :0.0780           Min.   :21.00  
##  1st Qu.:  0.0   1st Qu.:27.30   1st Qu.:0.2437           1st Qu.:24.00  
##  Median : 30.5   Median :32.00   Median :0.3725           Median :29.00  
##  Mean   : 79.8   Mean   :31.99   Mean   :0.4719           Mean   :33.24  
##  3rd Qu.:127.2   3rd Qu.:36.60   3rd Qu.:0.6262           3rd Qu.:41.00  
##  Max.   :846.0   Max.   :67.10   Max.   :2.4200           Max.   :81.00  
##  Outcome  
##  NO :500  
##  YES:268  
##           
##           
##           
## 
\end{verbatim}

\hypertarget{uni-variate-analysis}{%
\paragraph{\texorpdfstring{\textbf{Uni-variate
analysis}}{Uni-variate analysis}}\label{uni-variate-analysis}}

\begin{Shaded}
\begin{Highlighting}[]
\CommentTok{\#library(patchwork)}
\FunctionTok{par}\NormalTok{(}\AttributeTok{mfrow=}\FunctionTok{c}\NormalTok{(}\DecValTok{2}\NormalTok{,}\DecValTok{2}\NormalTok{))}
\NormalTok{p1 }\OtherTok{\textless{}{-}} \FunctionTok{hist}\NormalTok{(diabetes}\SpecialCharTok{$}\NormalTok{Pregnancies)}
\NormalTok{p2 }\OtherTok{\textless{}{-}} \FunctionTok{hist}\NormalTok{(diabetes}\SpecialCharTok{$}\NormalTok{Glucose)}
\NormalTok{p3 }\OtherTok{\textless{}{-}} \FunctionTok{hist}\NormalTok{(diabetes}\SpecialCharTok{$}\NormalTok{BMI)}
\NormalTok{p4 }\OtherTok{\textless{}{-}} \FunctionTok{hist}\NormalTok{(diabetes}\SpecialCharTok{$}\NormalTok{Age)}
\end{Highlighting}
\end{Shaded}

\includegraphics{Diabetes-project_files/figure-latex/unnamed-chunk-4-1.pdf}

From these distribution graphs, Age and number of times pregnant are not
in normal distributions as expected since the underlying population
should not be normally distributed either.

Glucose level and BMI are following a normal distribution.

\begin{Shaded}
\begin{Highlighting}[]
\FunctionTok{boxplot}\NormalTok{(diabetes}\SpecialCharTok{$}\NormalTok{BloodPressure, }\AttributeTok{ylab=}\StringTok{"BloodPressure"}\NormalTok{)}
\end{Highlighting}
\end{Shaded}

\includegraphics{Diabetes-project_files/figure-latex/unnamed-chunk-5-1.pdf}

\hypertarget{impact-of-glucose-on-diabetes}{%
\subsubsection{\texorpdfstring{\textbf{Impact of Glucose on
Diabetes}}{Impact of Glucose on Diabetes}}\label{impact-of-glucose-on-diabetes}}

\begin{Shaded}
\begin{Highlighting}[]
\FunctionTok{ggplot}\NormalTok{(diabetes, }\FunctionTok{aes}\NormalTok{(}\AttributeTok{x=}\NormalTok{Glucose))}\SpecialCharTok{+}
     \FunctionTok{geom\_histogram}\NormalTok{(}\AttributeTok{fill=}\StringTok{\textquotesingle{}deepskyblue\textquotesingle{}}\NormalTok{, }\AttributeTok{col=}\StringTok{\textquotesingle{}red\textquotesingle{}}\NormalTok{) }\SpecialCharTok{+}
      \FunctionTok{facet\_grid}\NormalTok{(Outcome}\SpecialCharTok{\textasciitilde{}}\NormalTok{.)}
\end{Highlighting}
\end{Shaded}

\includegraphics{Diabetes-project_files/figure-latex/unnamed-chunk-6-1.pdf}

Formulate a hypothesis to assess the mean difference of glucose levels
between the positive and negative groups.

\textbf{Conditions}

Individuals are independent of each other

Here distributions are skewed but the sample \textgreater30

Both the groups are independent of each other and the sample size is
lesser than 10\% of the population,

\begin{Shaded}
\begin{Highlighting}[]
\FunctionTok{library}\NormalTok{(report)}
\NormalTok{t }\OtherTok{\textless{}{-}} \FunctionTok{t.test}\NormalTok{(Glucose}\SpecialCharTok{\textasciitilde{}}\NormalTok{Outcome, }\AttributeTok{data =}\NormalTok{ diabetes) }
\NormalTok{t}
\end{Highlighting}
\end{Shaded}

\begin{verbatim}
## 
##  Welch Two Sample t-test
## 
## data:  Glucose by Outcome
## t = -13.752, df = 461.33, p-value < 2.2e-16
## alternative hypothesis: true difference in means between group NO and group YES is not equal to 0
## 95 percent confidence interval:
##  -35.74707 -26.80786
## sample estimates:
##  mean in group NO mean in group YES 
##          109.9800          141.2575
\end{verbatim}

\begin{Shaded}
\begin{Highlighting}[]
\FunctionTok{report}\NormalTok{(t)}
\end{Highlighting}
\end{Shaded}

\begin{verbatim}
## Effect sizes were labelled following Cohen's (1988) recommendations.
## 
## The Welch Two Sample t-test testing the difference of Glucose by Outcome (mean
## in group NO = 109.98, mean in group YES = 141.26) suggests that the effect is
## negative, statistically significant, and large (difference = -31.28, 95% CI
## [-35.75, -26.81], t(461.33) = -13.75, p < .001; Cohen's d = -1.28, 95% CI
## [-1.48, -1.08])
\end{verbatim}

p-value is \textless{} critical values of 0.05, so we reject the null
hypothesis for the alternate hypothesis. We can say that we are, 95 \%
confident, that the average glucose levels for individuals with diabetes
is \textgreater{} the people without diabetes.

\begin{Shaded}
\begin{Highlighting}[]
\NormalTok{t1 }\OtherTok{\textless{}{-}} \FunctionTok{t.test}\NormalTok{(Age}\SpecialCharTok{\textasciitilde{}}\NormalTok{Outcome, }\AttributeTok{data =}\NormalTok{ diabetes)}
\NormalTok{t1}
\end{Highlighting}
\end{Shaded}

\begin{verbatim}
## 
##  Welch Two Sample t-test
## 
## data:  Age by Outcome
## t = -6.9207, df = 575.78, p-value = 1.202e-11
## alternative hypothesis: true difference in means between group NO and group YES is not equal to 0
## 95 percent confidence interval:
##  -7.545092 -4.209236
## sample estimates:
##  mean in group NO mean in group YES 
##          31.19000          37.06716
\end{verbatim}

\begin{Shaded}
\begin{Highlighting}[]
\FunctionTok{report}\NormalTok{(t1)}
\end{Highlighting}
\end{Shaded}

\begin{verbatim}
## Effect sizes were labelled following Cohen's (1988) recommendations.
## 
## The Welch Two Sample t-test testing the difference of Age by Outcome (mean in
## group NO = 31.19, mean in group YES = 37.07) suggests that the effect is
## negative, statistically significant, and medium (difference = -5.88, 95% CI
## [-7.55, -4.21], t(575.78) = -6.92, p < .001; Cohen's d = -0.58, 95% CI [-0.74,
## -0.41])
\end{verbatim}

p\_value\textless0.001, suggest that diabetic people is mostly old.

\begin{Shaded}
\begin{Highlighting}[]
\FunctionTok{theme\_set}\NormalTok{(}\FunctionTok{theme\_test}\NormalTok{())}
\NormalTok{diabetes }\SpecialCharTok{\%\textgreater{}\%} 
  \FunctionTok{ggplot}\NormalTok{(}\FunctionTok{aes}\NormalTok{(}\AttributeTok{x=}\FunctionTok{cut}\NormalTok{(Age, }\AttributeTok{breaks =} \DecValTok{5}\NormalTok{))) }\SpecialCharTok{+}
        \FunctionTok{geom\_boxplot}\NormalTok{(}\FunctionTok{aes}\NormalTok{(}\AttributeTok{y=}\NormalTok{DiabetesPedigreeFunction), }\AttributeTok{col=}\StringTok{\textquotesingle{}deepskyblue\textquotesingle{}}\NormalTok{)}\SpecialCharTok{+}
  \FunctionTok{labs}\NormalTok{(}
    \AttributeTok{x =} \StringTok{"Age Breaks =5"}\NormalTok{,}
    \AttributeTok{y =} \StringTok{"DiabetesPedigreeFunction"}\NormalTok{,}
    \AttributeTok{colour =} \StringTok{" "}\NormalTok{,}
    \AttributeTok{shape =} \StringTok{" "}
\NormalTok{   ) }
\end{Highlighting}
\end{Shaded}

\includegraphics{Diabetes-project_files/figure-latex/unnamed-chunk-9-1.pdf}

\hypertarget{insulin-vs-glucose-based-on-outcome-as-diabetes}{%
\subsubsection{Insulin Vs Glucose based on Outcome as
diabetes}\label{insulin-vs-glucose-based-on-outcome-as-diabetes}}

\begin{Shaded}
\begin{Highlighting}[]
\NormalTok{diabetes }\SpecialCharTok{\%\textgreater{}\%} 
  \FunctionTok{ggplot}\NormalTok{(}\FunctionTok{aes}\NormalTok{(}\AttributeTok{x=}\NormalTok{Insulin,}\AttributeTok{y=}\NormalTok{Glucose)) }\SpecialCharTok{+}
    \FunctionTok{geom\_point}\NormalTok{() }\SpecialCharTok{+}
  \FunctionTok{geom\_smooth}\NormalTok{()}
\end{Highlighting}
\end{Shaded}

\includegraphics{Diabetes-project_files/figure-latex/unnamed-chunk-10-1.pdf}

\begin{Shaded}
\begin{Highlighting}[]
\CommentTok{\# par(mfrow=c(1,2))}
\CommentTok{\#boxplot}
\FunctionTok{library}\NormalTok{(patchwork)}
\NormalTok{p0 }\OtherTok{\textless{}{-}}\NormalTok{ diabetes }\SpecialCharTok{\%\textgreater{}\%} 
  \FunctionTok{ggplot}\NormalTok{(}\FunctionTok{aes}\NormalTok{(}\AttributeTok{x=}\NormalTok{DiabetesPedigreeFunction))}\SpecialCharTok{+}
  \FunctionTok{geom\_boxplot}\NormalTok{() }\SpecialCharTok{+}
  \FunctionTok{facet\_wrap}\NormalTok{(}\SpecialCharTok{\textasciitilde{}}\NormalTok{Outcome) }\SpecialCharTok{+} \FunctionTok{coord\_flip}\NormalTok{() }\SpecialCharTok{+}
  \FunctionTok{ggtitle}\NormalTok{(}\StringTok{\textquotesingle{}Boxplot\textquotesingle{}}\NormalTok{)}

\NormalTok{p1 }\OtherTok{\textless{}{-}}\NormalTok{ diabetes }\SpecialCharTok{\%\textgreater{}\%} 
  \FunctionTok{ggplot}\NormalTok{(}\FunctionTok{aes}\NormalTok{(}\AttributeTok{x=}\NormalTok{Glucose, }\AttributeTok{col=}\NormalTok{Outcome))}\SpecialCharTok{+}
    \FunctionTok{geom\_density}\NormalTok{()}\SpecialCharTok{+}
    \FunctionTok{facet\_wrap}\NormalTok{(}\SpecialCharTok{\textasciitilde{}}\NormalTok{Outcome) }\SpecialCharTok{+}
    \FunctionTok{ggtitle}\NormalTok{(}\StringTok{\textquotesingle{}Density plot of Glucose\textquotesingle{}}\NormalTok{)}

\NormalTok{(p0 }\SpecialCharTok{+}\NormalTok{ p1)}
\end{Highlighting}
\end{Shaded}

\includegraphics{Diabetes-project_files/figure-latex/unnamed-chunk-11-1.pdf}

From Density Plot the distribution is shifted towards the left for those
without diabetes.

This indicates those \textbf{without diabetes generally have a lower
blood glucose level}.

\begin{Shaded}
\begin{Highlighting}[]
\CommentTok{\#two sample t{-}test}
\NormalTok{t2 }\OtherTok{\textless{}{-}} \FunctionTok{t.test}\NormalTok{(DiabetesPedigreeFunction}\SpecialCharTok{\textasciitilde{}}\NormalTok{Outcome, }\AttributeTok{data =}\NormalTok{ diabetes) }\CommentTok{\#\%\textgreater{}\% report()}
\NormalTok{t2}
\end{Highlighting}
\end{Shaded}

\begin{verbatim}
## 
##  Welch Two Sample t-test
## 
## data:  DiabetesPedigreeFunction by Outcome
## t = -4.5768, df = 454.51, p-value = 6.1e-06
## alternative hypothesis: true difference in means between group NO and group YES is not equal to 0
## 95 percent confidence interval:
##  -0.17262065 -0.06891135
## sample estimates:
##  mean in group NO mean in group YES 
##          0.429734          0.550500
\end{verbatim}

\begin{Shaded}
\begin{Highlighting}[]
\FunctionTok{report}\NormalTok{(t2)}
\end{Highlighting}
\end{Shaded}

\begin{verbatim}
## Effect sizes were labelled following Cohen's (1988) recommendations.
## 
## The Welch Two Sample t-test testing the difference of DiabetesPedigreeFunction
## by Outcome (mean in group NO = 0.43, mean in group YES = 0.55) suggests that
## the effect is negative, statistically significant, and small (difference =
## -0.12, 95% CI [-0.17, -0.07], t(454.51) = -4.58, p < .001; Cohen's d = -0.43,
## 95% CI [-0.62, -0.24])
\end{verbatim}

\hypertarget{correlation-between-each-variable}{%
\paragraph{\texorpdfstring{\textbf{Correlation between each
variable}}{Correlation between each variable}}\label{correlation-between-each-variable}}

Scatter matrix of all columns

\begin{Shaded}
\begin{Highlighting}[]
\FunctionTok{library}\NormalTok{(GGally)}
\NormalTok{diabetes }\SpecialCharTok{\%\textgreater{}\%} \FunctionTok{select}\NormalTok{(}\SpecialCharTok{{-}}\NormalTok{Outcome) }\SpecialCharTok{\%\textgreater{}\%} 
  \FunctionTok{ggcorr}\NormalTok{(}
    \AttributeTok{name =} \StringTok{"corr"}\NormalTok{, }\AttributeTok{label =} \ConstantTok{TRUE}
\NormalTok{  ) }\SpecialCharTok{+}
  \FunctionTok{theme}\NormalTok{(}\AttributeTok{legend.position =} \StringTok{\textquotesingle{}none\textquotesingle{}}\NormalTok{) }\SpecialCharTok{+}
  \FunctionTok{labs}\NormalTok{(}\AttributeTok{title =} \StringTok{\textquotesingle{}Correlation Plot of Variance\textquotesingle{}}\NormalTok{) }\SpecialCharTok{+}
  \FunctionTok{theme}\NormalTok{(}\AttributeTok{plot.title =} \FunctionTok{element\_text}\NormalTok{(}
    \AttributeTok{face =} \StringTok{\textquotesingle{}bold\textquotesingle{}}\NormalTok{,                                }\AttributeTok{color =} \StringTok{\textquotesingle{}deepskyblue\textquotesingle{}}\NormalTok{,}
    \AttributeTok{hjust =} \FloatTok{0.5}\NormalTok{, }\AttributeTok{size =} \DecValTok{11}\NormalTok{)) }
\end{Highlighting}
\end{Shaded}

\includegraphics{Diabetes-project_files/figure-latex/unnamed-chunk-13-1.pdf}

Pregnancy, Age, Insulin, skinthickness are having higher correlation.

\hypertarget{fitting-a-logistic-regression-to-assess-importance-of-predictors}{%
\subsubsection{\texorpdfstring{\textbf{Fitting a logistic regression to
assess importance of
predictors}}{Fitting a logistic regression to assess importance of predictors}}\label{fitting-a-logistic-regression-to-assess-importance-of-predictors}}

\begin{itemize}
\item
  Fitting a GLM (General Linear Model) with link function `probit'
\item
  Target variable `diabetes' estimated to be binomially distributed
\end{itemize}

This is a generic implementation --- without assumption on data

\begin{Shaded}
\begin{Highlighting}[]
\NormalTok{logit }\OtherTok{\textless{}{-}} \FunctionTok{glm}\NormalTok{(Outcome}\SpecialCharTok{\textasciitilde{}}\NormalTok{.,, }\AttributeTok{data =}\NormalTok{ diabetes, }\AttributeTok{family =} \FunctionTok{binomial}\NormalTok{())}
\FunctionTok{summary}\NormalTok{(logit)}
\end{Highlighting}
\end{Shaded}

\begin{verbatim}
## 
## Call:
## glm(formula = Outcome ~ ., family = binomial(), data = diabetes)
## 
## Coefficients:
##                            Estimate Std. Error z value Pr(>|z|)    
## (Intercept)              -8.4046964  0.7166359 -11.728  < 2e-16 ***
## Pregnancies               0.1231823  0.0320776   3.840 0.000123 ***
## Glucose                   0.0351637  0.0037087   9.481  < 2e-16 ***
## BloodPressure            -0.0132955  0.0052336  -2.540 0.011072 *  
## SkinThickness             0.0006190  0.0068994   0.090 0.928515    
## Insulin                  -0.0011917  0.0009012  -1.322 0.186065    
## BMI                       0.0897010  0.0150876   5.945 2.76e-09 ***
## DiabetesPedigreeFunction  0.9451797  0.2991475   3.160 0.001580 ** 
## Age                       0.0148690  0.0093348   1.593 0.111192    
## ---
## Signif. codes:  0 '***' 0.001 '**' 0.01 '*' 0.05 '.' 0.1 ' ' 1
## 
## (Dispersion parameter for binomial family taken to be 1)
## 
##     Null deviance: 993.48  on 767  degrees of freedom
## Residual deviance: 723.45  on 759  degrees of freedom
## AIC: 741.45
## 
## Number of Fisher Scoring iterations: 5
\end{verbatim}

\begin{Shaded}
\begin{Highlighting}[]
\FunctionTok{report}\NormalTok{(logit)}
\end{Highlighting}
\end{Shaded}

\begin{verbatim}
## We fitted a logistic model (estimated using ML) to predict Outcome with
## Pregnancies, Glucose, BloodPressure, SkinThickness, Insulin, BMI,
## DiabetesPedigreeFunction and Age (formula: Outcome ~ Pregnancies + Glucose +
## BloodPressure + SkinThickness + Insulin + BMI + DiabetesPedigreeFunction +
## Age). The model's explanatory power is substantial (Tjur's R2 = 0.33). The
## model's intercept, corresponding to Pregnancies = 0, Glucose = 0, BloodPressure
## = 0, SkinThickness = 0, Insulin = 0, BMI = 0, DiabetesPedigreeFunction = 0 and
## Age = 0, is at -8.40 (95% CI [-9.86, -7.05], p < .001). Within this model:
## 
##   - The effect of Pregnancies is statistically significant and positive (beta =
## 0.12, 95% CI [0.06, 0.19], p < .001; Std. beta = 0.42, 95% CI [0.21, 0.63])
##   - The effect of Glucose is statistically significant and positive (beta = 0.04,
## 95% CI [0.03, 0.04], p < .001; Std. beta = 1.12, 95% CI [0.90, 1.36])
##   - The effect of BloodPressure is statistically significant and negative (beta =
## -0.01, 95% CI [-0.02, -3.10e-03], p = 0.011; Std. beta = -0.26, 95% CI [-0.46,
## -0.06])
##   - The effect of SkinThickness is statistically non-significant and positive
## (beta = 6.19e-04, 95% CI [-0.01, 0.01], p = 0.929; Std. beta = 9.87e-03, 95% CI
## [-0.20, 0.23])
##   - The effect of Insulin is statistically non-significant and negative (beta =
## -1.19e-03, 95% CI [-2.97e-03, 5.82e-04], p = 0.186; Std. beta = -0.14, 95% CI
## [-0.34, 0.07])
##   - The effect of BMI is statistically significant and positive (beta = 0.09, 95%
## CI [0.06, 0.12], p < .001; Std. beta = 0.71, 95% CI [0.48, 0.95])
##   - The effect of DiabetesPedigreeFunction is statistically significant and
## positive (beta = 0.95, 95% CI [0.37, 1.54], p = 0.002; Std. beta = 0.31, 95% CI
## [0.12, 0.51])
##   - The effect of Age is statistically non-significant and positive (beta = 0.01,
## 95% CI [-3.50e-03, 0.03], p = 0.111; Std. beta = 0.17, 95% CI [-0.04, 0.39])
## 
## Standardized parameters were obtained by fitting the model on a standardized
## version of the dataset. 95% Confidence Intervals (CIs) and p-values were
## computed using a Wald z-distribution approximation.
\end{verbatim}

\hypertarget{filtering-the-most-important-predictors-from-glm-model-extracting-the-n-most-important-glm-coefficients}{%
\paragraph{Filtering the most important predictors from GLM model
Extracting the N most important GLM
coefficients}\label{filtering-the-most-important-predictors-from-glm-model-extracting-the-n-most-important-glm-coefficients}}

\hypertarget{features-selection}{%
\subsubsection{Features Selection}\label{features-selection}}

\begin{itemize}
\tightlist
\item
  Highest logistic model coefficients
\end{itemize}

\begin{Shaded}
\begin{Highlighting}[]
\NormalTok{model\_coef }\OtherTok{\textless{}{-}} \FunctionTok{exp}\NormalTok{(}\FunctionTok{coef}\NormalTok{(logit))[}\DecValTok{2}\SpecialCharTok{:}\FunctionTok{ncol}\NormalTok{(diabetes)]}
\NormalTok{model\_coef }\OtherTok{\textless{}{-}}\NormalTok{ model\_coef[}\FunctionTok{c}\NormalTok{(}\FunctionTok{order}\NormalTok{(model\_coef,}\AttributeTok{decreasing =} \ConstantTok{TRUE}\NormalTok{)[}\DecValTok{1}\SpecialCharTok{:}\NormalTok{(}\FunctionTok{ncol}\NormalTok{(diabetes)}\SpecialCharTok{{-}}\DecValTok{1}\NormalTok{)])]}
\NormalTok{predictors\_names }\OtherTok{\textless{}{-}} \FunctionTok{c}\NormalTok{(}\FunctionTok{names}\NormalTok{(model\_coef), }\FunctionTok{names}\NormalTok{(diabetes)[}\FunctionTok{length}\NormalTok{(diabetes)])}
\NormalTok{predictors\_names}
\end{Highlighting}
\end{Shaded}

\begin{verbatim}
## [1] "DiabetesPedigreeFunction" "Pregnancies"             
## [3] "BMI"                      "Glucose"                 
## [5] "Age"                      "SkinThickness"           
## [7] "Insulin"                  "BloodPressure"           
## [9] "Outcome"
\end{verbatim}

\begin{Shaded}
\begin{Highlighting}[]
\CommentTok{\#filter df with most important predictors}
\NormalTok{diabetes\_df }\OtherTok{\textless{}{-}}\NormalTok{ diabetes[, }\FunctionTok{c}\NormalTok{(predictors\_names)]}
\FunctionTok{head}\NormalTok{(diabetes\_df)}
\end{Highlighting}
\end{Shaded}

\begin{verbatim}
## # A tibble: 6 x 9
##   DiabetesPedigreeFunction Pregnancies   BMI Glucose   Age SkinThickness Insulin
##                      <dbl>       <dbl> <dbl>   <dbl> <dbl>         <dbl>   <dbl>
## 1                    0.627           6  33.6     148    50            35       0
## 2                    0.351           1  26.6      85    31            29       0
## 3                    0.672           8  23.3     183    32             0       0
## 4                    0.167           1  28.1      89    21            23      94
## 5                    2.29            0  43.1     137    33            35     168
## 6                    0.201           5  25.6     116    30             0       0
## # i 2 more variables: BloodPressure <dbl>, Outcome <fct>
\end{verbatim}

\hypertarget{outlier-detection}{%
\subsubsection{Outlier Detection}\label{outlier-detection}}

\begin{Shaded}
\begin{Highlighting}[]
\FunctionTok{library}\NormalTok{(DMwR2)}
\NormalTok{outlier\_scores }\OtherTok{\textless{}{-}}\NormalTok{ diabetes }\SpecialCharTok{\%\textgreater{}\%} \FunctionTok{select}\NormalTok{(}\SpecialCharTok{{-}}\NormalTok{Outcome) }\SpecialCharTok{\%\textgreater{}\%} 
            \FunctionTok{lofactor}\NormalTok{(}\AttributeTok{k=}\DecValTok{5}\NormalTok{)}
\FunctionTok{plot}\NormalTok{(}\FunctionTok{density}\NormalTok{(outlier\_scores))}
\end{Highlighting}
\end{Shaded}

\includegraphics{Diabetes-project_files/figure-latex/unnamed-chunk-17-1.pdf}

\begin{Shaded}
\begin{Highlighting}[]
\NormalTok{outliers }\OtherTok{\textless{}{-}} \FunctionTok{order}\NormalTok{(outlier\_scores, }
                  \AttributeTok{decreasing =} \ConstantTok{TRUE}\NormalTok{)[}\DecValTok{1}\SpecialCharTok{:}\DecValTok{5}\NormalTok{]}
\FunctionTok{print}\NormalTok{(outliers)}
\end{Highlighting}
\end{Shaded}

\begin{verbatim}
## [1]  14 580 146 454 107
\end{verbatim}

\hypertarget{the-five-outliers-obtained-in-the-output-are-the-row-numbers-in-the-diabetes1-data-derived-from-the-diabetes-data-set.}{%
\subsection{\texorpdfstring{\textbf{The five outliers obtained in the
output are the row numbers in the diabetes1 data derived from the
diabetes data
set.}}{The five outliers obtained in the output are the row numbers in the diabetes1 data derived from the diabetes data set.}}\label{the-five-outliers-obtained-in-the-output-are-the-row-numbers-in-the-diabetes1-data-derived-from-the-diabetes-data-set.}}

\begin{Shaded}
\begin{Highlighting}[]
\CommentTok{\# \#labels outliers}
\NormalTok{n }\OtherTok{\textless{}{-}} \FunctionTok{nrow}\NormalTok{(diabetes)}
\NormalTok{labels }\OtherTok{\textless{}{-}} \DecValTok{1}\SpecialCharTok{:}\NormalTok{n}
\NormalTok{labels[}\SpecialCharTok{{-}}\NormalTok{outliers] }\OtherTok{\textless{}{-}} \StringTok{"."}
\FunctionTok{biplot}\NormalTok{(}\FunctionTok{prcomp}\NormalTok{(diabetes[,}\SpecialCharTok{{-}}\DecValTok{9}\NormalTok{], }\AttributeTok{na.rm=}\ConstantTok{TRUE}\NormalTok{), }\AttributeTok{cex=}\NormalTok{.}\DecValTok{8}\NormalTok{, }\AttributeTok{xlabs=}\NormalTok{labels)}
\end{Highlighting}
\end{Shaded}

\includegraphics{Diabetes-project_files/figure-latex/unnamed-chunk-19-1.pdf}

\begin{Shaded}
\begin{Highlighting}[]
\FunctionTok{library}\NormalTok{(Rlof)}
\NormalTok{outlier.scores }\OtherTok{\textless{}{-}} \FunctionTok{lof}\NormalTok{(diabetes[,}\SpecialCharTok{{-}}\DecValTok{9}\NormalTok{], }\AttributeTok{k=}\DecValTok{5}\NormalTok{)}
\NormalTok{outlier.scores}\OtherTok{\textless{}{-}}\FunctionTok{lof}\NormalTok{(diabetes[,}\SpecialCharTok{{-}}\DecValTok{9}\NormalTok{],}\AttributeTok{k=}\FunctionTok{c}\NormalTok{(}\DecValTok{5}\SpecialCharTok{:}\DecValTok{10}\NormalTok{))}
\NormalTok{outlier.scores }\SpecialCharTok{\%\textgreater{}\%} \FunctionTok{head}\NormalTok{(}\DecValTok{4}\NormalTok{)}
\end{Highlighting}
\end{Shaded}

\begin{verbatim}
##             5        6        7         8         9        10
## [1,] 1.061391 1.029842 1.046721 1.0418848 1.0389602 1.0533413
## [2,] 1.070537 1.051352 1.034477 1.0391998 0.9969303 1.0016080
## [3,] 1.078872 1.115526 1.153996 1.1710755 1.1498865 1.1508931
## [4,] 1.030767 1.024495 1.031342 0.9995388 0.9937247 0.9915746
\end{verbatim}

\hypertarget{data-modelling}{%
\subsection{Data Modelling}\label{data-modelling}}

\begin{Shaded}
\begin{Highlighting}[]
\CommentTok{\#data partitioning/Train and Test}
\FunctionTok{library}\NormalTok{(rsample)}
\NormalTok{split }\OtherTok{\textless{}{-}} \FunctionTok{initial\_split}\NormalTok{(diabetes, }\AttributeTok{prop =} \DecValTok{8}\SpecialCharTok{/}\DecValTok{10}\NormalTok{)}
\NormalTok{train }\OtherTok{\textless{}{-}} \FunctionTok{training}\NormalTok{(split)}
\NormalTok{test }\OtherTok{\textless{}{-}} \FunctionTok{testing}\NormalTok{(split)}
\end{Highlighting}
\end{Shaded}

\begin{enumerate}
\def\labelenumi{\arabic{enumi}.}
\item
  \textbf{Basic GLM with all Variables}

\begin{Shaded}
\begin{Highlighting}[]
\NormalTok{log }\OtherTok{\textless{}{-}} \FunctionTok{glm}\NormalTok{(Outcome}\SpecialCharTok{\textasciitilde{}}\NormalTok{., }\AttributeTok{family =} \FunctionTok{binomial}\NormalTok{(), }\AttributeTok{data =}\NormalTok{ train)}
\FunctionTok{summary}\NormalTok{(log)}
\end{Highlighting}
\end{Shaded}

\begin{verbatim}
## 
## Call:
## glm(formula = Outcome ~ ., family = binomial(), data = train)
## 
## Coefficients:
##                            Estimate Std. Error z value Pr(>|z|)    
## (Intercept)              -7.5854963  0.7549097 -10.048  < 2e-16 ***
## Pregnancies               0.1409007  0.0346813   4.063 4.85e-05 ***
## Glucose                   0.0314221  0.0039780   7.899 2.81e-15 ***
## BloodPressure            -0.0156997  0.0058319  -2.692   0.0071 ** 
## SkinThickness            -0.0005365  0.0074219  -0.072   0.9424    
## Insulin                  -0.0006313  0.0009384  -0.673   0.5011    
## BMI                       0.0790267  0.0158781   4.977 6.45e-07 ***
## DiabetesPedigreeFunction  0.8002574  0.3182116   2.515   0.0119 *  
## Age                       0.0181326  0.0098239   1.846   0.0649 .  
## ---
## Signif. codes:  0 '***' 0.001 '**' 0.01 '*' 0.05 '.' 0.1 ' ' 1
## 
## (Dispersion parameter for binomial family taken to be 1)
## 
##     Null deviance: 795.19  on 613  degrees of freedom
## Residual deviance: 597.26  on 605  degrees of freedom
## AIC: 615.26
## 
## Number of Fisher Scoring iterations: 5
\end{verbatim}
\end{enumerate}

The result shows that the variables Triceps\_Skin, Serum\_Insulin and
Age are not statistically significant.~p\_values is \textgreater{} 0.01
so we can experiment by removing it.

\hypertarget{logistic-model}{%
\subsection{\texorpdfstring{\textbf{Logistic
Model}}{Logistic Model}}\label{logistic-model}}

input: explanatory variables xk and provides a prediction p with
parameters \(βk\).

The logit transformation constrains the value of p to the interval {[}0,
1{]}.

\#

βk represents the log-odds of feature xk says how much the logarithm of
the odds of a positive outcome (i.e.~the logit transform) increases when
predictor xk increases by .

Likelihood of the model as follows:

\#

\(Y\^i\) = outcome of subject i.

Maximizing the likelihood = maximizing the log-likelihood(model)

\#

The above equation is non-linear for logistic regression and its
minimization is generally done numerically by iteratively re-weighted
least-squares

\begin{Shaded}
\begin{Highlighting}[]
\NormalTok{smodel }\OtherTok{\textless{}{-}} \FunctionTok{step}\NormalTok{(log)}
\end{Highlighting}
\end{Shaded}

\begin{verbatim}
## Start:  AIC=615.26
## Outcome ~ Pregnancies + Glucose + BloodPressure + SkinThickness + 
##     Insulin + BMI + DiabetesPedigreeFunction + Age
## 
##                            Df Deviance    AIC
## - SkinThickness             1   597.27 613.27
## - Insulin                   1   597.71 613.71
## <none>                          597.26 615.26
## - Age                       1   600.66 616.66
## - DiabetesPedigreeFunction  1   603.80 619.80
## - BloodPressure             1   604.69 620.69
## - Pregnancies               1   614.51 630.51
## - BMI                       1   625.32 641.32
## - Glucose                   1   674.21 690.21
## 
## Step:  AIC=613.27
## Outcome ~ Pregnancies + Glucose + BloodPressure + Insulin + BMI + 
##     DiabetesPedigreeFunction + Age
## 
##                            Df Deviance    AIC
## - Insulin                   1   597.86 611.86
## <none>                          597.27 613.27
## - Age                       1   600.70 614.70
## - DiabetesPedigreeFunction  1   603.83 617.83
## - BloodPressure             1   604.92 618.92
## - Pregnancies               1   614.52 628.52
## - BMI                       1   628.17 642.17
## - Glucose                   1   676.00 690.00
## 
## Step:  AIC=611.86
## Outcome ~ Pregnancies + Glucose + BloodPressure + BMI + DiabetesPedigreeFunction + 
##     Age
## 
##                            Df Deviance    AIC
## <none>                          597.86 611.86
## - Age                       1   601.64 613.64
## - DiabetesPedigreeFunction  1   604.03 616.03
## - BloodPressure             1   605.66 617.66
## - Pregnancies               1   615.38 627.38
## - BMI                       1   628.19 640.19
## - Glucose                   1   680.25 692.25
\end{verbatim}

\hypertarget{the-final-model-is-chosen-with-aic-as-the-selection-generated-from-a-logistic-regression-model-with-the-lowest-aic-value-of-584.68.}{%
\subsubsection{\texorpdfstring{\textbf{The final model is chosen with
AIC as the selection generated from a logistic regression model with the
lowest AIC value of
584.68.}}{The final model is chosen with AIC as the selection generated from a logistic regression model with the lowest AIC value of 584.68.}}\label{the-final-model-is-chosen-with-aic-as-the-selection-generated-from-a-logistic-regression-model-with-the-lowest-aic-value-of-584.68.}}

\hypertarget{initial-parameter-tuning}{%
\subsection{\texorpdfstring{\textbf{Initial Parameter
Tuning}}{Initial Parameter Tuning}}\label{initial-parameter-tuning}}

\begin{Shaded}
\begin{Highlighting}[]
\FunctionTok{library}\NormalTok{(rpart)}
\FunctionTok{library}\NormalTok{(rpart.plot)}
\NormalTok{tree }\OtherTok{\textless{}{-}} \FunctionTok{rpart}\NormalTok{(Outcome}\SpecialCharTok{\textasciitilde{}}\NormalTok{., }\AttributeTok{method =} \StringTok{\textquotesingle{}class\textquotesingle{}}\NormalTok{, }\AttributeTok{data =}\NormalTok{ diabetes)}
\FunctionTok{rpart.plot}\NormalTok{(tree)}
\end{Highlighting}
\end{Shaded}

\includegraphics{Diabetes-project_files/figure-latex/unnamed-chunk-24-1.pdf}

\begin{Shaded}
\begin{Highlighting}[]
\FunctionTok{plotcp}\NormalTok{(tree)}
\end{Highlighting}
\end{Shaded}

\includegraphics{Diabetes-project_files/figure-latex/unnamed-chunk-25-1.pdf}

\textbf{Complexity parameter}

The above tree was tuned using a reference of the Relative error VS
Complexity parameter. From the above figure the Cp value of 0.016, the
decision tree was pruned. The final decision tree

\begin{Shaded}
\begin{Highlighting}[]
\NormalTok{tree1 }\OtherTok{\textless{}{-}} \FunctionTok{rpart}\NormalTok{(Outcome}\SpecialCharTok{\textasciitilde{}}\NormalTok{., }\AttributeTok{method =} \StringTok{\textquotesingle{}class\textquotesingle{}}\NormalTok{, }\AttributeTok{data =}\NormalTok{ diabetes, }\AttributeTok{cp=}\FloatTok{0.016}\NormalTok{)}
\FunctionTok{rpart.plot}\NormalTok{(tree1)}
\end{Highlighting}
\end{Shaded}

\includegraphics{Diabetes-project_files/figure-latex/unnamed-chunk-26-1.pdf}

If CP value is lower, tree will grow large. A cp = 1 will provide no
tree which helps in pruning the tree. Higher complexity parameters can
lead to an over pruned tree.

\textbf{2nd Model By removing 3 features-}

\begin{Shaded}
\begin{Highlighting}[]
\NormalTok{log1 }\OtherTok{\textless{}{-}} \FunctionTok{glm}\NormalTok{(Outcome}\SpecialCharTok{\textasciitilde{}}\NormalTok{., }\AttributeTok{family =} \FunctionTok{binomial}\NormalTok{(), }\AttributeTok{data=}\NormalTok{(train }\SpecialCharTok{\%\textgreater{}\%} \FunctionTok{select}\NormalTok{(}\SpecialCharTok{{-}}\FunctionTok{c}\NormalTok{(Age,SkinThickness, }
\NormalTok{                BloodPressure ,                                         Insulin))))}

\FunctionTok{summary}\NormalTok{(log1)}
\end{Highlighting}
\end{Shaded}

\begin{verbatim}
## 
## Call:
## glm(formula = Outcome ~ ., family = binomial(), data = (train %>% 
##     select(-c(Age, SkinThickness, BloodPressure, Insulin))))
## 
## Coefficients:
##                           Estimate Std. Error z value Pr(>|z|)    
## (Intercept)              -7.692497   0.692235 -11.113  < 2e-16 ***
## Pregnancies               0.158119   0.029620   5.338 9.39e-08 ***
## Glucose                   0.031024   0.003626   8.557  < 2e-16 ***
## BMI                       0.065313   0.014330   4.558 5.17e-06 ***
## DiabetesPedigreeFunction  0.806612   0.309695   2.605   0.0092 ** 
## ---
## Signif. codes:  0 '***' 0.001 '**' 0.01 '*' 0.05 '.' 0.1 ' ' 1
## 
## (Dispersion parameter for binomial family taken to be 1)
## 
##     Null deviance: 795.19  on 613  degrees of freedom
## Residual deviance: 608.07  on 609  degrees of freedom
## AIC: 618.07
## 
## Number of Fisher Scoring iterations: 4
\end{verbatim}

\begin{Shaded}
\begin{Highlighting}[]
\FunctionTok{report}\NormalTok{(log1)}
\end{Highlighting}
\end{Shaded}

\begin{verbatim}
## We fitted a logistic model (estimated using ML) to predict Outcome with
## Pregnancies, Glucose, BMI and DiabetesPedigreeFunction (formula: Outcome ~
## Pregnancies + Glucose + BMI + DiabetesPedigreeFunction). The model's
## explanatory power is substantial (Tjur's R2 = 0.29). The model's intercept,
## corresponding to Pregnancies = 0, Glucose = 0, BMI = 0 and
## DiabetesPedigreeFunction = 0, is at -7.69 (95% CI [-9.10, -6.39], p < .001).
## Within this model:
## 
##   - The effect of Pregnancies is statistically significant and positive (beta =
## 0.16, 95% CI [0.10, 0.22], p < .001; Std. beta = 0.54, 95% CI [0.35, 0.74])
##   - The effect of Glucose is statistically significant and positive (beta = 0.03,
## 95% CI [0.02, 0.04], p < .001; Std. beta = 0.98, 95% CI [0.76, 1.21])
##   - The effect of BMI is statistically significant and positive (beta = 0.07, 95%
## CI [0.04, 0.09], p < .001; Std. beta = 0.53, 95% CI [0.31, 0.76])
##   - The effect of DiabetesPedigreeFunction is statistically significant and
## positive (beta = 0.81, 95% CI [0.21, 1.42], p = 0.009; Std. beta = 0.28, 95% CI
## [0.07, 0.49])
## 
## Standardized parameters were obtained by fitting the model on a standardized
## version of the dataset. 95% Confidence Intervals (CIs) and p-values were
## computed using a Wald z-distribution approximation.
\end{verbatim}

\begin{Shaded}
\begin{Highlighting}[]
\FunctionTok{par}\NormalTok{(}\AttributeTok{mfrow=}\FunctionTok{c}\NormalTok{(}\DecValTok{2}\NormalTok{,}\DecValTok{2}\NormalTok{))}
\FunctionTok{plot}\NormalTok{(log1)}
\end{Highlighting}
\end{Shaded}

\includegraphics{Diabetes-project_files/figure-latex/unnamed-chunk-28-1.pdf}

1. \textbf{Residuals vs fitted values}; Here dotted line at y=0
indicates fit line. The points on fit line indication of zero residual.
Points above are having positive residuals similarly points below have
negative residuals. . The red line is indicates smoothed high order
polynomial curve which provides the idea behind pattern of residual
movements. Here the residuals have logarithmic pattern hence we got a
good model.

2. \textbf{Normal Q-Q Plot:} In general Normal Q-Q plot is used to check
if our residuals follow Normal distribution or not. The residuals are
said to be \textbf{normally distributed} if points follow the dotted
line closely.

In our case residual points follow the dotted line closely except for
observation at 229, 350 and 503 So this model residuals passed the test
of Normality.

3. \textbf{Scale --- Location Plot:} It indicates spread of points
across predicted values range.

Assumption:

- Variance should be reasonably equal across the predictor
range(Homoscedasticity)

So this horizontal red line is set to be ideal and it indicates that
residuals have uniform variance across the Predictor range. As residuals
spread wider from each other the red spread line goes up. In this case
the data is Homoscedastic i.e having \textbf{uniform variance}.

4. \textbf{Residuals vs Leverage Plot}:

\textbf{Influence}: The Influence of an observation can be defined in
terms of how much the predicted scores would change if the observation
is excluded. Cook's Distance

\textbf{Leverage}: The leverage of an observation is defined on how much
the observation's value on the predictor variable differs from the mean
of the predictor variable. \textbf{The more the leverage of an
observation , the greater potential that point has in terms of
influence}.

In our plot the dotted red lines are the cook's distance and the areas
of interest for us are the ones outside the dotted line on the top right
corner or bottom right corner. If any point falls in that region, we say
the observation has high leverage or having some potential for
influencing our model is higher if we exclude that point.

\textbf{3rd Model: Predict Diabetes Risk on new patients using Decision
Tree}

\begin{Shaded}
\begin{Highlighting}[]
\FunctionTok{library}\NormalTok{(party)}
\end{Highlighting}
\end{Shaded}

\begin{verbatim}
## Loading required package: grid
\end{verbatim}

\begin{verbatim}
## Loading required package: mvtnorm
\end{verbatim}

\begin{verbatim}
## Loading required package: modeltools
\end{verbatim}

\begin{verbatim}
## Loading required package: stats4
\end{verbatim}

\begin{verbatim}
## Loading required package: strucchange
\end{verbatim}

\begin{verbatim}
## Loading required package: zoo
\end{verbatim}

\begin{verbatim}
## 
## Attaching package: 'zoo'
\end{verbatim}

\begin{verbatim}
## The following objects are masked from 'package:base':
## 
##     as.Date, as.Date.numeric
\end{verbatim}

\begin{verbatim}
## Loading required package: sandwich
\end{verbatim}

\begin{verbatim}
## 
## Attaching package: 'strucchange'
\end{verbatim}

\begin{verbatim}
## The following object is masked from 'package:stringr':
## 
##     boundary
\end{verbatim}

\begin{verbatim}
## 
## Attaching package: 'party'
\end{verbatim}

\begin{verbatim}
## The following object is masked from 'package:dplyr':
## 
##     where
\end{verbatim}

\begin{Shaded}
\begin{Highlighting}[]
\NormalTok{ct }\OtherTok{\textless{}{-}} \FunctionTok{ctree}\NormalTok{(Outcome}\SpecialCharTok{\textasciitilde{}}\NormalTok{.,}\AttributeTok{data=}\NormalTok{train)}
\CommentTok{\# plot(ct)}
\NormalTok{predict\_clas }\OtherTok{\textless{}{-}} \FunctionTok{predict}\NormalTok{(ct, test,}
                         \AttributeTok{type=}\FunctionTok{c}\NormalTok{(}\StringTok{\textquotesingle{}response\textquotesingle{}}\NormalTok{))}
\FunctionTok{table}\NormalTok{(predict\_clas, test}\SpecialCharTok{$}\NormalTok{Outcome)}
\end{Highlighting}
\end{Shaded}

\begin{verbatim}
##             
## predict_clas NO YES
##          NO  94  28
##          YES  7  25
\end{verbatim}

\begin{Shaded}
\begin{Highlighting}[]
\FunctionTok{library}\NormalTok{(caret)}
\end{Highlighting}
\end{Shaded}

\begin{verbatim}
## Loading required package: lattice
\end{verbatim}

\begin{verbatim}
## 
## Attaching package: 'caret'
\end{verbatim}

\begin{verbatim}
## The following object is masked from 'package:purrr':
## 
##     lift
\end{verbatim}

\begin{Shaded}
\begin{Highlighting}[]
\NormalTok{con\_mat }\OtherTok{\textless{}{-}} \FunctionTok{confusionMatrix}\NormalTok{(test}\SpecialCharTok{$}\NormalTok{Outcome, predict\_clas, }\AttributeTok{positive =} \ConstantTok{NULL}\NormalTok{,}
              \AttributeTok{dnn =} \FunctionTok{c}\NormalTok{(}\StringTok{\textquotesingle{}Prediction\textquotesingle{}}\NormalTok{, }\StringTok{\textquotesingle{}References\textquotesingle{}}\NormalTok{))}
\NormalTok{con\_mat}
\end{Highlighting}
\end{Shaded}

\begin{verbatim}
## Confusion Matrix and Statistics
## 
##           References
## Prediction NO YES
##        NO  94   7
##        YES 28  25
##                                           
##                Accuracy : 0.7727          
##                  95% CI : (0.6984, 0.8363)
##     No Information Rate : 0.7922          
##     P-Value [Acc > NIR] : 0.7596165       
##                                           
##                   Kappa : 0.4442          
##                                           
##  Mcnemar's Test P-Value : 0.0007232       
##                                           
##             Sensitivity : 0.7705          
##             Specificity : 0.7812          
##          Pos Pred Value : 0.9307          
##          Neg Pred Value : 0.4717          
##              Prevalence : 0.7922          
##          Detection Rate : 0.6104          
##    Detection Prevalence : 0.6558          
##       Balanced Accuracy : 0.7759          
##                                           
##        'Positive' Class : NO              
## 
\end{verbatim}

\hypertarget{th-model-nauxefve-bayes}{%
\subsubsection{4th Model Naïve Bayes:}\label{th-model-nauxefve-bayes}}

\begin{Shaded}
\begin{Highlighting}[]
\FunctionTok{library}\NormalTok{(e1071)}
\NormalTok{nb }\OtherTok{\textless{}{-}} \FunctionTok{naiveBayes}\NormalTok{(Outcome}\SpecialCharTok{\textasciitilde{}}\NormalTok{., }\AttributeTok{data =}\NormalTok{ train)}
\NormalTok{pred }\OtherTok{\textless{}{-}} \FunctionTok{predict}\NormalTok{(nb, test)}
\FunctionTok{confusionMatrix}\NormalTok{(test}\SpecialCharTok{$}\NormalTok{Outcome, pred)}
\end{Highlighting}
\end{Shaded}

\begin{verbatim}
## Confusion Matrix and Statistics
## 
##           Reference
## Prediction NO YES
##        NO  91  10
##        YES 20  33
##                                           
##                Accuracy : 0.8052          
##                  95% CI : (0.7337, 0.8645)
##     No Information Rate : 0.7208          
##     P-Value [Acc > NIR] : 0.01045         
##                                           
##                   Kappa : 0.5482          
##                                           
##  Mcnemar's Test P-Value : 0.10035         
##                                           
##             Sensitivity : 0.8198          
##             Specificity : 0.7674          
##          Pos Pred Value : 0.9010          
##          Neg Pred Value : 0.6226          
##              Prevalence : 0.7208          
##          Detection Rate : 0.5909          
##    Detection Prevalence : 0.6558          
##       Balanced Accuracy : 0.7936          
##                                           
##        'Positive' Class : NO              
## 
\end{verbatim}

\textbf{Though it's a basic model still it performed well with 77\%
accuracy on an average}

\end{document}
